\documentclass[12pt,a4paper,draft]{article}
\usepackage[ngerman]{babel}
\usepackage[T1]{fontenc}
\usepackage[utf8]{inputenc}

\newcommand{\unsername}{Code for Niederrhein}

\renewcommand{\labelenumi}{(\arabic{enumi})}
\renewcommand{\labelenumii}{\arabic{enumii}.}

\title{Beitragsordnung des Vereins \unsername}
\author{\unsername}
\date{\today}

\begin{document}

\maketitle
\tableofcontents

\section{Allgemeines}
\begin{enumerate}
\item Diese Beitragsordnung ist nicht Bestandteil der Satzung. Sie regelt die 
Beitragsverpflichtungen der Mitglieder sowie die Gebühren und Umlagen. Sie kann 
nur von der Mitgliederversammlung des Vereins geändert werden.

\item Die Mitgliederversammlung beschließt die Höhe der Geldbeiträge, die von 
Mitgliedern erhoben werden.

\item Von ordentlichen Mitgliedern und Fördermitgliedern werden Geldbeiträge 
erhoben.

\item Von Ehrenmitgliedern werden keine Geldbeiträge erhoben.
\end{enumerate}

\section{Regelmäßige Beiträge}
\begin{enumerate}
\item Von ordentlichen Mitgliedern wird ein jährlicher Mitgliedsbeitrag von 60 
EUR erhoben.

\item Von ordentlichen Mitgliedern, die die Schule besuchen, Rente oder 
Sozialleistungen empfangen, oder studieren, wird auf Antrag ein veringerter 
jährlicher Mitgliedsbeitrag von 5 EUR erhoben. Der Anspruch ist jährlich zu 
belegen.

\item Von Fördermitgliedern wird ein jährlicher Mitgliedsbeitrag von mindestens 
60 EUR erhoben.

\item Die Mitgliedsbeiträge sind jeweils zu Beginn des Jahres binnen zehn 
Werktagen in voller Höhe zu zahlen.

\item Bei Erwerb der Mitgliedschaft ist der Mitgliedsbeitrag für das laufende 
Kalenderjahr anteilhaft, binnen fünf Werktagen zu zahlen.

\item Bei Beendigung der Mitgliedschaft werden keine Mitgliedsbetärge erstattet.
\end{enumerate}

\section{Zahlung der Beiträge}
\begin{enumerate}
\item Geldbeiträge sind auf das Vereinskonto zu überweisen oder dem Vorstand 
bar zu entrichten.

\item Andere Zahlungsweisen sind nicht zulässig.
\end{enumerate}

\end{document}