\documentclass[12pt,a4paper,draft]{article}
\usepackage[ngerman]{babel}
\usepackage[T1]{fontenc}
\usepackage[utf8]{inputenc}

\newcommand{\unsername}{Code for Niederrhein}

\renewcommand{\labelenumi}{(\arabic{enumi})}
\renewcommand{\labelenumii}{\arabic{enumii}.}

\title{Satzung des Vereins \unsername} % §25 BGB
\author{\unsername}
\date{\today}

\begin{document}

\maketitle
\tableofcontents

\section{Name, Eintragung, Sitz, Geschäftsjahr}
\begin{enumerate}
\item Der Verein führt den Namen `\unsername'.

\item Er soll in das Vereinsregister eingetragen werden und trägt dann den 
Zusatz `e.V.'

\item Der Sitz des Vereins ist Moers. % §24 BGB

\item Geschäftsjahr ist das Kalenderjahr.
\end{enumerate}

\section{Vereinszweck} % §57 BGB
\begin{enumerate}
\item Der Zweck des Vereins ist die Förderung der Volks- und Berufsbildung, die 
Förderung von Wissenschaft und Forschung und die Förderung von Kunst und 
Kultur. %Siehe https://www.gesetze-im-internet.de/ao_1977/__52.html

\item Der Zweck wird insbesondere verwirklicht durch:

\begin{enumerate}
\item Pflege und Intensivierung des Erfahrungs- und Informationsaustausches zu 
Themen Demokratie und Bürgerbeteiligung, Politik- und Verwaltungstransparenz, 
offene und freie Kommunikations- und Informationstechnologie und digitale 
Bildung.

\item Vorbereitung, Durchführung oder Förderung von gemeinsamen Forschungs- und 
Lernprojekten und Bildungsveranstaltungen.

\item Die Veranstaltung von Kongressen, Treffen und Konferenzen.
\end{enumerate}

\item Der Verein verfolgt ausschließlich und unmittelbar gemeinnützige Zwecke 
im Sinne des Abschnitts `Steuerbegünstigte Zwecke' der Abgabenordnung.

\item Der Verein ist selbstlos tätig; er verfolgt nicht in erster Linie 
eigenwirtschaftliche Zwecke.
\end{enumerate}

\section{Mitgliedschaft}
\subsection{Arten der Mitgliedschaft}
\begin{enumerate}
\item Mitglieder des Vereins sind ordentliche Mitglieder, Ehrenmitglieder und 
Fördermitglieder, die den oben genannten Vereinszweck ideell oder materiell 
unterstützen.

\item Fördermitglieder können sowohl natürliche als auch juristische Personen 
sein.

\item Ordentliche Mitglieder und Ehrenmitglieder können nur natürliche Personen 
sein.

\item Ordentliche Mitglieder und Ehrenmitglieder haben das Recht, an der 
Mitgliederversammlung des Vereins teilzunehmen, Anträge zu stellen, und das 
Stimmrecht auszuüben.

\item Der Vorstand kann der Mitgliederversammlung die Ernennung von 
Ehrenmitgliedern vorschlagen. Ehrenmitglieder sind von Beitragszahlungen 
freigestellt.
\end{enumerate}

\subsection{Erwerb der Mitgliedschaft} % §58 1. BGB
\begin{enumerate}
\item Über die Aufnahme weiterer Mitglieder entscheidet der Vorstand.

\item Der Beitrittsantrag hat den Namen, die vollständige Adresse sowie eine 
gültige E-Mail-Adresse zu enthalten.

\item Gegen die Ablehnung, die keiner Begründung bedarf, steht dem/der 
Bewerber/in die Berufung an die Mitgliederversammlung zu, welche dann 
endgültig entscheidet.

\item Das aufgenommene Mitglied erhält eine Kopie der Satzung. Die jeweils 
aktuelle Satzung wird darüber hinaus an geeigneter Stelle den Mitgliedern 
verfügbar gemacht.

\item Eine Ehrenmitgliedschaft kann im Einverständnis zwischen dem Mitglied 
und dem Vortstand in eine ordentliche Mitgliedschaft umgewandelt werden.

\item Zum Erwerb der ordentlichen Mitgliedschaft oder Umwandlung der 
Ehrenmitgliedschaft in eine ordentliche Mitgliedschaft ist bei 
Minderjährigen die Zustimmung des gesetzlichen Vertreters erforderlich.
\end{enumerate}

\subsection{Beendigung der Mitgliedschaft} % §58 1. BGB
\begin{enumerate}
\item Die Mitgliedschaft endet durch Austritt, Ausschluss, Tod, oder durch 
Auflösung der juristischen Person. Ein Anspruch auf Rückerstattung bereits 
gezahlter Mitgliedsbeiträge besteht nicht.

\item Ein Ausschluss kann nur aus wichtigem Grund erfolgen. 
Wichtige Gründe sind:

\begin{enumerate}
\item Ein die Vereinsziele schädigendes Verhalten.

\item Die Verletzung satzungsmäßiger Pflichten.

\item Straffälligkeit.
 
\item Nicht nachkommen der Zahlungsverpflichtung trotz trotz erfolgter 
einfacher Mahnung bei einer Fristsetzung von drei Wochen
\end{enumerate}

\item Über den Ausschluss entscheidet der Vorstand.

\item Gegen den Ausschluss steht dem Mitglied die Berufung an die 
Mitgliederversammlung zu, die schriftlich binnen eines Monats an den Vorstand 
zu richten ist.

\item Die Mitgliederversammlung entscheidet im Rahmen des Vereins endgültig.

\item Der Austritt aus dem Verein kann jederzeit ohne Wahrung einer Frist 
gegenüber einem vertretungsberechtigten Vorstandsmitglied schriftlich erklärt 
werden. % §39 BGB
\end{enumerate}

\section{Organe des Vereins}

\subsection{Mitgliederversammlung} % §58 4. BGB
\begin{enumerate}
\item Die Mitgliederversammlung ist das oberste Vereinsorgan. Zu ihren Aufgaben 
gehören insbesondere:

\begin{enumerate}
\item Wahl und Abwahl des Vorstands. % §27 (1) BGB

\item Entlastung des Vorstands.

\item Entgegennahme der Berichte des Vorstandes.

\item Wahl der Kassenprüfern/innen.

\item Festsetzung von Beiträgen und deren Fälligkeit.

\item Beschlussfassung über die Änderung der Satzung.

\item Beschlussfassung über die Auflösung des Vereins.

\item Entscheidung über Aufnahme und Ausschluss von Mitgliedern in 
Berufungsfällen.

\item Beschlussfassung über die Beitragsordnung.

\item Weitere Aufgaben, soweit sich diese aus der Satzung oder nach dem Gesetz 
ergeben.
\end{enumerate}

\item Die Mitgliederversammlung entlastet den Vorstand nach Entgegennahme des 
jährlich schriftlich vorzulegenden Geschäftsberichts / Jahresberichts des 
Vorstandes und des Prüfungsberichts des Kassenprüfers.

\item Der Mitgliederversammlung gehören alle stimmberechtigten 
Vereinsmitglieder mit je einer Stimme an.

\item Das Stimmrecht kann nur persönlich oder für ein Mitglied unter Vorlage 
einer schriftlichen Vollmacht ausgeübt werden.

\item Bei Abstimmungen entscheidet die einfache Mehrheit der abgegebenen 
Stimmen.

\item Abstimmungen erfolgen öffentlich, sofern nicht von mindestens einem 
Abstimmungsberechtigten eine geheime Abstimmung gewünscht wird.

\item Die Leitung der Versammlung hat ein Mitglied des Vorstands oder ein von 
der Mitgliederversammlung bestimmter Versammlungsleiter.

\item Zu Beginn der Mitgliederversammlung ist ein Schriftführer zu wählen.

\item Über die Beschlüsse der Mitgliederversammlung ist ein Protokoll 
anzufertigen, das vom Versammlungsleiter und dem Schriftführer zu unterzeichnen 
ist.

%\item Das Protokoll ist nach Unterzeichnung allgemein offenzulegen.

\item In jedem Geschäftsjahr findet mindestens eine ordentliche 
Mitgliederversammlung statt.

\item Die Mitgliederversammlung wird vom Vorstand schriftlich oder in 
elektronischer Form als E-Mail unter Angabe der Tagesordnung und unter 
Einhaltung einer Frist von einem Monat einberufen. % §32 (1) BGB

\item Die Tagesordnung ist zu ergänzen, wenn dies ein Mitglied bis spätestens 
eine Woche vor dem angesetzten Termin schriftlich beantragt. Die Ergänzung 
ist zu Beginn der Versammlung bekanntzumachen.

\item Wahlen und Abwahlen von Vorstandsmitgliedern und Änderungen dieser 
Satzung bedürfen der ausdrücklichen Nennung in der Tagesordnung, mit der 
eingeladen wird.

\item Der Vorstand ist zur Einberufung einer außerordentlichen 
Mitgliederversammlung verpflichtet, wenn mindestens ein Drittel der Mitglieder 
dies schriftlich unter Angabe von Gründen verlangt. % §§ 36, 37 BGB

\item Die Mitgliederversammlung ist bei ordnungsgemäßer Einladung 
beschlussfähig ohne Rücksicht auf die Anzahl der Erschienenen.

\item Satzungsänderungen und die Auflösung des Vereins können nur mit einer 
Mehrheit von drei Vierteln der anwesenden Mitglieder beschlossen werden. 
% §33 (1) BGB

\item Stimmenthaltungen und ungültige Stimmen bleiben außer Betracht.
\end{enumerate}

\subsection{Vorstand} % §§ 26, 58 3. BGB
\begin{enumerate}
\item Der Vorstand setzt sich aus drei Mitgliedern zusammen. % §30 BGB

\item Der Verein wird gerichtlich und außergerichtlich von zwei 
Vorstandsmitgliedern gemeinsam vertreten.

\item Der Vorstand steht den Mitgliedern für Vorschläge zur Verfügung, 
entscheidet über diese und setzt sie satzungsgemäß um.

\item Der Vorstand beschließt über alle Vereinsangelegenheiten, soweit sie 
nicht eines Beschlusses der Mitgliederversammlung bedürfen. Er führt die 
Beschlüsse der Mitgliederversammlung aus.

\item Beschlüsse des Vorstands können auch schriftlich oder fernmündlich 
gefasst werden. Schriftlich oder fernmündlich gefasste Vorstandsbeschlüsse 
sind schriftlich niederzulegen und vom Vorstand zu unterzeichnen.

\item Der Vorstand ist berechtigt außerordentliche Mitgliederversammlungen 
einzuberufen, sofern es zur Erfüllung des Vereinszwecks erforderlich ist.

\item Der Vorstand wird von der Mitgliederversammlung für die Dauer eines 
Jahres gewählt.

\item Vorstandsmitglieder können nur ordentliche Mitglieder und Ehrenmitglieder 
des Vereins werden.

\item Wiederwahl ist zulässig.

\item Der Vorstand bleibt solange im Amt, bis ein neuer Vorstand gewählt ist.

\item Bei Beendigung der Mitgliedschaft im Verein endet auch das Amt als 
Vorstand.

\item Scheidet ein Vorstandsmitglied während der Amtszeit aus, so haben die 
übrigen Vorstandsmitglieder eine Ergänzung herbeizuführen, die der Bestätigung 
durch dienächste Mitgliederversammlung bedarf.

\item Der Vorstand arbeitet ehrenamtlich. % §27 (3) BGB
\end{enumerate}

\section{Auflösung des Vereins}
\begin{enumerate}
\item Der Verein kann durch Beschluss der Mitgliederversammlung aufgelöst 
werden. Zu dem Beschluss ist eine Mehrheit von drei Vierteln der abgegebenen 
Stimmen erforderlich. % §41 BGB

\item Bei Auflösung des Vereins oder Wegfall Steuerbegünstigter Zwecke fällt 
das Vermögen des Vereins an den Open Knowedge Foundation Deutschland e.V., der 
es unmittelbar und ausschließlich für gemeinnützige oder mildtätige Zweke zu 
verwenden hat. % §45 (1) BGB
\end{enumerate}

\section{Finanz- und Beitragsordnung} % §58 2. BGB
\begin{enumerate}
\item Von den Mitgliedern werden Geldbeiträge erhoben.

\item Die Höhe dieser Zahlungen, die Fälligkeit, die Art und Weise der Zahlung 
und zusätzliche Gebühren bei Zahlungsverzug oder Verwendung eines anderen als 
des beschlossenen Zahlungsverfahrens regelt eine Beitragsordnung.

\item Die Beitragsordnung wird von der Mitgliederversammlung beschlossen.

\item Die Beitragsordnung ist nicht Satzungsbestandteil.

\item Die Beitragsordnung wird den Mitgliedern in der jeweils aktuellen Fassung 
bekanntgegeben.

\item Verantwortlich für die finanzielle Tätigkeit des Vereins ist der 
Vorstand.

\item Für Rechtsgeschäfte mit einem Geschäftswert bis 500,00 EUR netto ist 
jedes Vorstandsmitglied einzeln zur gerichtlichen und außergerichtlichen 
Vertretung des Vereins berechtigt. Für andere Geschäfte ist die gemeinsame 
Vertretung durch zwei Vorstandsmitglieder erforderlich.

\item Im Rahmen der Vorstandssitzungen erstattet der Vorstand Bericht über die 
aktuelle finanzielle Situation des Vereins.

\item Die Mittel des Vereins dürfen nur für satzungsgemäße Zwecke verwendet 
werden. Es darf keine Person durch Ausgaben, die dem Vereinszweck fremd sind, 
oder durch unverhältnismäßig hohe Vergütungen begünstigt werden. Die Mitglieder 
erhalten keine Zuwendungen aus den Mitteln des Vereins.

\item Ausscheidende Mitglieder haben keinen Anspruch auf das Vereinsvermögen.

\item Vom Verein bereitgestellte Mittel sind nach dem Ausscheiden 
unaufgefordert innerhalb von 14 Tagen an den Vorstand auszuhändigen.

\item Für anfallende Aufgaben und Arbeiten, die das zumutbare Maß 
ehrenamtlicher Tätigkeiten übersteigt, können Personen beschäftigt werden. Es 
dürfen dafür keine unverhältnismäßig hohen Vergütungen gewährt werden.

\item Jeder, der im Namen des Vereins Gelder einnimmt oder ausgibt, hat dies 
ordentlich zu dokumentieren. Hierzu gehören Datum, Art der Einnahme/Ausgabe und 
Betrag.

\item Auslagen werden nur gegen Einreichung von Belegen erstattet.

\item Die Mitgliederversammlung wählt für die Dauer von einem Jahr eine/n 
Kassenprüfer/in.

\item Die/Der Kassenprüfer/in darf nicht Mitglied des Vorstands sein.

\item Die Wiederwahl der Kassenprüferin/des Kassenprüfers ist zulässig.
\end{enumerate}

\section{Schlussbestimmung}
\begin{enumerate}
\item Der Vorstand ist befugt redaktionelle Änderungen an dieser Satzung 
durchzuführen, sofern sie einer Auflage des Registergerichtes oder einer 
Behörde entsprechen müssen.

\item Diese Satzung tritt mit der Eintragung in das Vereinsregister in Kraft.
\end{enumerate}

% Zur Eintragung:
% 7 Unterschriften nach §56 BGB
% Urkunde über die Vorstandswahl

\end{document}
